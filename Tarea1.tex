% Created 2022-09-17 Sat 21:47
% Intended LaTeX compiler: pdflatex
\documentclass{homework}
\usepackage[utf8]{inputenc}
\usepackage[T1]{fontenc}
\usepackage{graphicx}
\usepackage{longtable}
\usepackage{wrapfig}
\usepackage{rotating}
\usepackage[normalem]{ulem}
\usepackage{amsmath}
\usepackage{amssymb}
\usepackage{capt-of}
\usepackage{hyperref}
\usepackage{amssymb, amsmath}
\usepackage{lastpage}
\usepackage[spanish]{babel}
\usepackage{eulervm}
\usepackage{concrete}
\usepackage{tikz-cd}
\usetikzlibrary{babel}
\pagenumbering{arabic}
\usepackage{microtype}
\rol{202103004-2}
\logo{usm.jpg}
\ayudante{Agustin Huerta}
\class{MAT125: Profesor Alexander Quaas}
\author{Jorge Eduardo Bravo Soto}
\date{\today}
\title{Tarea 1}
\hypersetup{
 pdfauthor={Jorge Eduardo Bravo Soto},
 pdftitle={Tarea 1},
 pdfkeywords={},
 pdfsubject={},
 pdfcreator={Emacs 28.1 (Org mode 9.6)}, 
 pdflang={Spanish}}
\begin{document}

\maketitle

\question Sean \(X\) e \(Y\) conjuntos, Demostrar que \(X = Y \iff (X \cap Y^{c}) \cup (X^{c} \cap Y) = \emptyset\) y \((X - Y) \cup (Y - X) = (X \cup Y) - (Y \cap X)\)
\label{sec:org1ee97d9}
\parte
Sea \(X = Y\) entonces
\((X \inter Y^{c}) \union (X^{c} \inter Y) = (X \inter X^{c}) \union (X^{c} \inter X) = \emptyset \union \emptyset = \emptyset\)
Por lo que queda demostrada la implicancia hacia la derecha.

Para demostrar la implicancia hacia la izquierda ocuparemos demostración por contradicción por lo que asumiremos que
\(X \neq Y\) y que \((X \inter Y^{c}) \cup (X^{c} \cap Y) = \emptyset\). Dado que \(X \neq Y\) existe, sin perdida
de generalidad, algún \(y_0 \in Y \land y_0 \notin X\) por tanto dado que \(y_0 \notin X \implies y_0 \in X^{c}\)
sabemos que \(y_0 \in (X^{c} \cap Y)\) por lo que \((X \cap Y^{c}) \cup (X^{c} \cap Y) \neq \emptyset\)
una contradicción. El caso en el que existe algún \(x_0 \in X \land x_0 \notin Y\) es análogo.

\parte
Ocuparemos solo las definiciones, distributividad de operadores lógicos, Morgan y llegaremos a la igualdad
\begin{align*}
&x \in (X - Y) \cup (Y - X)\\
&\iff  (x \in X \land x \notin Y) \lor (x \in Y \land x \notin X)\\
&\iff ((x \in X \land x \notin Y) \lor x \in Y) \land ((x \in X \land x \notin Y) \lor x \notin X)\\
&\iff ((x \in X \lor x \in Y) \land (x \notin Y \lor x \in Y)) \land ((x \in X \lor x \notin X) \land (x \notin Y \lor x \notin X))\\
&\iff ((x \in X \lor x \in Y) \land T) \land (T \land (x \notin Y \lor x \notin X))\\
&\iff (x \in X \lor x \in Y) \land (x \notin Y \lor x \notin X)\\
&\iff x \in (X \cup Y) \land x \in (Y^{c} \cup X^{c})\\
&\iff x \in (X \cup Y) \land x \in (Y \cap X)^{c}\\
&\iff x \in (X \cup Y) - (Y \cap X)
\end{align*}

\question Sea \(f : A \to B\), sean \(X, Y \subset A\), demostrar que \(f(X) - f(Y) \subset f(X - Y)\) y \(f(X) - f(Y) = f(X - Y)\) si f es inyectiva
\label{sec:org366fa32}
\parte
\begin{align*}
    x \in f(X) - f(Y) &\iff x \in f(X) \land x \notin f(Y)\\
&\iff \exists k \in X, \forall u \in Y, f(k) = x \land  f(u) \neq x\\
&\implies k \in X \land k \notin Y \implies x \in f(X - Y)
\end{align*}

\parte
Sea \(f\) inyectiva también
\begin{align*}
    &x \in f(X - Y) \iff \exists k \in X - Y, f(k) = x\\
    &\iff k \in X \land k \notin Y, f(k) = x
\end{align*}

por la inyectividad sabemos que \(k\) es el único valor en el Dominio
tal que \(f(p) = x\) por lo que podemos decir que dado
que \(k \notin Y\), para todo \(u\) en \(Y\) no se cumple que \(f(u) = x\)
\begin{align*}
    &\implies \forall u \in Y, k \in X, f(k) = x \land f(u) \neq x\\
    &\iff x \in f(X) \land x \notin f(Y)\\
    &\iff x \in f(X) - f(Y)
\end{align*}
\question Demostrar que \(f^{-1}(V)\) con \(V \subset Y\) no es ambiguo si f es biyectiva
\label{sec:orgb90e391}
\parte
Notemos que dado que \(f\) es una biyeccion \(f(x) = y \iff x = f^{-1}(y)\)

\begin{align*}
    & x \in PreIm(f(V)) \iff \exists y \in V, f(x) = y\\
    & \iff y \in V, x = f^{-1}(y)\\
    \intertext{Por definicion de imagen}
    & \iff x \in Im(f^{-1}(V))
\end{align*}

por tanto \(x \in PreIm(f(V)) \iff x \in Im(f^{-1}(V))\)

\question Demostrar ciertas propiedades de la cardinalidad sobre conjuntos finitos
\label{sec:org20334e1}
\parte
Consideraremos que \(x_0 \notin X\) y demostraremos que \(|X \cup \{x_0\}| = |X| + 1\).
Sea \(|X| = n\), dado que este es finito existe una biyeccion entre
\(X\) e \(I_n\), llamaremos a esta biyeccion \(\varphi(x)\)

Consideremos la función definida por
\begin{equation*}
    f : X \to I_{n+1}
    \begin{cases}
        f(x) = \varphi(x) & x \neq x_0\\
        f(x) = n + 1 & x = x_0
    \end{cases}
\end{equation*}

Demostraremos que esto es una biyeccion.
\(f\) es sobreyectiva ya que \(f(X) = \varphi(X) = I_n\) y \(f(x_0) = n + 1\)
por lo que para cada valor en \(I_{n+1}\) existe al menos una imagen.

Ahora demostraremos inyectividad, si \(f(x) = f(y)\) y a su vez \(f(x) \leq n\)
nos queda que \(\varphi(x) = \varphi(y)\) y dado que \(\varphi\) es inyectiva nos queda
\(x = y\), en caso de que \(f(x) = n + 1\) la única preimagen que tiene es \(x_0\)
por lo que \(f(x) = f(y) \iff x = x_0 = y\) y queda demostrado la inyectividad.

\parte
Sea \(f(x)\) la biyeccion de \(X\) en \(I_n\) y \(g(x)\) la biyeccion de \(Y\) en \(I_m\)
Consideremos la siguiente función
\begin{equation*}
    h(x) = \begin{cases}
        f(x) & x \in X\\
        n + g(x) & x \notin X \land x \in Y
    \end{cases}
\end{equation*}

Notemos que para todo \(x\) se cumple que
\begin{equation*}
    f(x) < n + g(x)
\end{equation*}
y por tricotomía tenemos entonces que
\begin{equation*}
    f(x) \neq n + g(x)
\end{equation*}

Demostraremos que \(h\) es inyectiva
\begin{align*}
    h(x) &= h(y)\\
    f(x) = f(y) &\lor n + g(x) = n + g(y)\\
    x = y &\lor g(x) = g(y)\\
    x = y &\lor x = y\\
    x &= y
\end{align*}

por tanto \(|X \cup Y| \leq |X| + |Y|\) y a su vez es finito.

Ahora supongamos que \(X \cap Y = \emptyset\) entonces demostraremos que \(h\) es sobreyectiva.
dado un \(x_0 \in I_{n + m}\) tenemos que si \(x_0 \leq n\) entonces existe \(x \in X\) tal que \(f(x) = x_0\) lo que implica que \(h(x) = f(x) = x_0\) dado que \(x \in X\).
ahora si tenemos que \(n + 1 \leq x_0 \leq n + m\), sabemos que \(x_0 = n + x_1\) tal que \(x_1 \in I_m\) entonces existe \(x \in Y\) tal que \(g(x) = x_1 \iff n + g(x) = x_0 = n + x_1\)
sabemos que ese \(x\) existe ya que \(g\) es sobreyectiva en \(I_m\) y \(x_1 \in I_m\), dado que la intersección es vacía \(x \notin X\)
por tanto \(h(x) = n + x_1 = x_0\), lo que significa que \(h\) es sobreyectiva. Por tanto \(|X \cup Y| = |X| + |Y|\) si \(X \cap Y = \emptyset\)

\parte
Sea \(f: X \to I_n\) una biyeccion y \(g: Y \to I_m\) otra biyección, notar que si consideramos
la restricción de \(f\) a \(X \cap Y\) esta sigue siendo inyectiva, por lo que \(|X \cap Y| \leq |X| + |Y|\)
Por tanto es finito.

Ocuparemos un proceso inductivo en el tamaño de \(X \cap Y\), notar que para \(X \cap Y = \emptyset\) es el problema anterior por lo que el caso base queda listo. si agregamos un elemento \(h\) a \(X \cap Y\) este
elemento se encuentra en ambos por tanto esta en la unión, por otro lado tenemos lo siguiente

\begin{equation*}
    |X \cup Y \cup \{h\}| + |(X \cap Y) \cup \{h\}| = |X \cup Y| + 1 + |X \cap Y| + 1 = |X \cup Y| + |X \cap Y| + 2
\end{equation*}

Esto es cierto por la parte a de este problema. despues obtenemos por Hipotesis de nuestro proceso inductivo
\begin{equation*}
    |X \cup Y| + |X \cap Y| + 2 = |X| + |Y| + 2 = |X| + 1 + |Y| + 1
\end{equation*}

por ultimo por la parte a de nuevo tenemos
\begin{equation*}
    |X \cup \{h\}| + |Y \cup \{h\}|
\end{equation*}

repitiendo \(n\) veces el proceso sigue que \(|X \cup Y| + |X \cap Y| = |X| + |Y|\)
\parte
Sea \(X\) e \(Y\) finitos, tal que \(|X| = m\), sea \(f\) la biyeccion de \(X\) en \(I_m\), haremos inducción sobre el tamaño de |Y|.
Si |Y| = 1 entonces, existe solo 1 elemento \(y_0 \in Y\), entonces todo elemento de \(X \times Y\) es de la
forma \((x, y_0)\), consideraremos la biyeccion trivial de \(\varphi : X \times Y \to I_m\) dada por \(\varphi((x, y)) = f(x)\)

\begin{align*}
    \varphi(x, y) &= \varphi(a, b)\\
    f(x) &= f(a)\\
    x &= a
\end{align*}
y dado que la segunda coordenada solo puede tomar el valor de \(y_0\) tenemos que es inyectiva. Dado un elemento \(k \in I_m\), \(\varphi(f^{-1}(k), y_0) = k\)
por tanto biyectiva. y \(|X \times Y| = m \cdot 1 = |X| \cdot |Y|\)

Paso inductivo asumamos que \(|Y| = n + 1\), agregándole el elemento y\textsubscript{n+1} entonces
\begin{equation*}
    X \times Y = (X \times (Y - \{y_{n+1}\})) \cup (X \times \{y_{n+1}\})
\end{equation*}
Dado que la segunda coordenada es distinta para elementos en \(X \times (Y - \{y_{n+1}\})\) y \(X \times \{y_{n+1}\}\), estos son disjuntos
por lo que podemos aplicar la parte B para decir que su cardinalidad es la suma de las cardinalidades.
\begin{equation*}
    |X \times (Y - \{y_{n+1}\}) \cup X \times \{y_{n+1}\}| = |X \times (Y - \{y_{n+1}\})| + |X \times \{y_{n+1}\}|
\end{equation*}
y por hipotesis de inductiva obtenemos
\begin{equation*}
    |X \times (Y - \{y_{n+1}\})| + |X \times \{y_{n+1}\}| = m \cdot n + m = m \cdot (n + 1)
\end{equation*}
la ultima igualdad por definición de la multiplicación, despues de aplicar \(n\) veces el proceso obtenemos el resultado esperado.

\question Demostrar que tener el mismo cardinal es una relacion de equivalencia y que satisface ciertas propiedades
\label{sec:org570dccc}
\parte
Notar que \(f = \sig{id}{X}{X}\) es una biyeccion de \(X\) en \(X\)

Directamente desde la definición de \(f\)
\begin{align*}
    f(x) &= f(y)\\
    x &= y
\end{align*}
Por tanto inyectiva

Dado un \(x \in X\) entonces \(f(x) = x\) por tanto es sobreyectiva. De esto sigue que es
biyectiva. Lo que significa que \(X\) tiene el mismo cardinal que \(X\)

\parte
Si \(X\) tiene el mismo cardinal que \(Y\) entonces existe
una función \(f: X \to Y\) que es biyectiva, sabemos
que \(f^{-1}: Y \to X\) es biyectiva por tanto \(Y\) tiene el mismo cardinal que \(X\)

\parte
Si \(X\) tiene el mismo cardinal que \(Y\) e \(Y\) tiene el mismo cardinal que \(Z\)
entonces sabemos que existen biyecciones \(\varphi : X \to Y\) y \(\psi : Y \to Z\).
Sabemos que la composición de funciones biyectivas es biyectiva por tanto
\(f : X \to Z\) tal que \(f(x) = \psi(\varphi(x))\) es biyectiva. Por tanto
\(X\) tiene el mismo cardinal que \(Z\)

\parte
Consideremos una función inyectiva \(\varphi : X \to Y\) la cual existe ya que \(X\) tiene menor cardinal que
\(Y\), ahora dado que \(\varphi\) es inyectiva tiene un inversa por la izquierda que llamaremos \(\psi : Y \to X\).
Pero si \(\psi \circ \varphi = id_X\) tenemos que \(\psi\) tiene una inversa por la derecha (\(\varphi\)) por lo cual es sobreyectiva.
Lo que significa que \(Y\) tiene mayor cardinal que \(X\).

\parte
Consideremos una función sobreyectiva de \(\sig{\phi}{X}{Y}\) la cual existe ya que \(X\) tiene mayor cardinal
que \(Y\), ahora dado que \(\varphi\) es sobreyectiva tiene una inversa por la derecha tal que \(\sig{\psi}{Y}{X}\)
y \(\varphi(\psi(y)) = y\), pero esto significa que \(\varphi \circ \psi = id_Y\) es decir que \(\psi\) tiene inversa
por la izquierda (\(\varphi\)) por lo cual es inyectiva. Lo que significa que \(Y\) tiene menor cardinal que \(X\)


\question test
\label{sec:org08c61ec}
\parte
\end{document}
